\documentclass[a4paper]{article}

%% Language and font encodings
\usepackage[english]{babel}
\usepackage[utf8x]{inputenc}
\usepackage[T1]{fontenc}

%% Sets page size and margins
\usepackage[a4paper,top=3cm,bottom=2cm,left=3cm,right=3cm,marginparwidth=1.75cm]{geometry}

%% Useful packages
\usepackage{amsmath}
\usepackage{graphicx}
\usepackage[colorinlistoftodos]{todonotes}
\usepackage[colorlinks=true, allcolors=blue]{hyperref}

\begin{document}

\section{Información básica}
\textbf{Fecha y hora: 11/01/2018 18:00}
\newline
\textbf{Participantes:}
\begin{itemize}
	\item Fernando Luján Martínez.
    \item Gregorio Díaz Descalzo.
    \item Enrique Brazález (a partir de las 17:40).
\end{itemize}
\section{Orden del día}
\begin{itemize}
\item Reunión inicial del trabajo.
\item ver como vamos a organizarlo todo.
\end{itemize}
\section{Desarrollo}
Después de navidad nos reunimos para comenzar de lleno con el trabajo. Hablamos sobre que herramientas vamos a utilizar, dejando claro que el flujo lo modelaremos con mule, los patrones Medit4CEP, utilizaremos un repositorio de github y para la gestión en general Visual Studio Team Services. También hablamos sobre la documentación que tenemos que realizar, primero tenemos que hacer el anteproyecto y segundo la memoria. Por último, hablamos sobre como vamos a tratar con los de Ingeteam y que queremos empezar ya para que todos nos pongamos a trabajar.
\section{Conclusiones}
\begin{itemize}
\item Dejamos definidas todas las herramientas y plazos para la siguiente reunión.
\item Reunión con Pedro de Ingeteam para meter un poco de prisa.
\item Poner en marcha todos los entornos y herramientas que hemos definido.
\end{itemize}

\end{document}




