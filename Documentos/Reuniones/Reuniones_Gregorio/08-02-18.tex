\documentclass[a4paper]{article}

%% Language and font encodings
\usepackage[english]{babel}
\usepackage[utf8x]{inputenc}
\usepackage[T1]{fontenc}

%% Sets page size and margins
\usepackage[a4paper,top=3cm,bottom=2cm,left=3cm,right=3cm,marginparwidth=1.75cm]{geometry}

%% Useful packages
\usepackage{amsmath}
\usepackage{graphicx}
\usepackage[colorinlistoftodos]{todonotes}
\usepackage[colorlinks=true, allcolors=blue]{hyperref}

\begin{document}

\section{Información básica}
\textbf{Fecha y hora: 08/02/2018 17:00}
\newline
\textbf{Participantes:}
\begin{itemize}
	\item Fernando Luján Martínez.
    \item Gregorio Díaz Descalzo.
    \item Enrique Brazález.
\end{itemize}
\section{Orden del día}
\begin{itemize}
\item Explicación de como hacer el anteproyecto.
\item Metodología a utilizar.
\end{itemize}
\section{Desarrollo}
Gregorio comenzó a explicarlos como estaba estructurado el documento y en la parte de competencias que debíamos poner en cada una de ellas. También nos indicó donde estaban las plantillas. Después hablamos sobre que metodología íbamos a escoger, desde el principio nos decantamos por una ágil, finalmente escogiendo scrum junto con prototipado.
\section{Conclusiones}
\begin{itemize}
\item Comenzar con el anteproyecto.
\end{itemize}

\end{document}