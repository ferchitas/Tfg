\documentclass[a4paper]{article}

%% Language and font encodings
\usepackage[english]{babel}
\usepackage[utf8x]{inputenc}
\usepackage[T1]{fontenc}

%% Sets page size and margins
\usepackage[a4paper,top=3cm,bottom=2cm,left=3cm,right=3cm,marginparwidth=1.75cm]{geometry}

%% Useful packages
\usepackage{amsmath}
\usepackage{graphicx}
\usepackage[colorinlistoftodos]{todonotes}
\usepackage[colorlinks=true, allcolors=blue]{hyperref}

\begin{document}

\section{Información básica}
\textbf{Fecha y hora: 15-02-2018 17:50}
\newline
\textbf{Participantes:}
\begin{itemize}
	\item Fernando Luján Martínez.
    \item Gregorio Díaz Descalzo.
    \item Enrique Brazález.
\end{itemize}
\section{Orden del día}
\begin{itemize}
\item Estado general del proyecto.
\item Revisión de cómo hay que hacer la memoria.
\item Retrospectiva del spring de Enrique.

\end{itemize}
\section{Desarrollo}
Le explicamos como llevamos el proyecto Enrique y yo, lo que tenemos hecho, los problemas que nos han surgido, algunos relacionados con el personal de Ingeteam ya que son un poco dejados en el trato.
Gregorio nos indica algunos consejos, en general, como tenemos que ir haciendo la memoria del trabajo. Sobre la introducción, el resumen, organización de los capítulos.
Sobre la retrospectiva del Enrique se comenta que la haga el solo que es quien mejor sabrá como debe de ir todo.

\section{Conclusiones}
\begin{itemize}
\item Terminar el anteproyecto.
\item Pedir excels y logs de fallos a Antonio.
\item Comenzar con la memoria.
\end{itemize}

\end{document}