\documentclass[a4paper]{article}

%% Language and font encodings
\usepackage[english]{babel}
\usepackage[utf8x]{inputenc}
\usepackage[T1]{fontenc}

%% Sets page size and margins
\usepackage[a4paper,top=3cm,bottom=2cm,left=3cm,right=3cm,marginparwidth=1.75cm]{geometry}

%% Useful packages
\usepackage{amsmath}
\usepackage{graphicx}
\usepackage[colorinlistoftodos]{todonotes}
\usepackage[colorlinks=true, allcolors=blue]{hyperref}

\begin{document}

\section{Información básica}
\textbf{Fecha y hora: 22/02/2018 17:20}
\newline
\textbf{Participantes:}
\begin{itemize}
	\item Fernando Luján Martínez.
    \item Gregorio Díaz Descalzo.
    \item Enrique Brazález.
\end{itemize}
\section{Orden del día}
\begin{itemize}
\item Revisión de proyecto.
\item Revisión del anteproyecto
\item Revisión del estado de los excel que tengo.
\end{itemize}
\section{Desarrollo}
Comenzamos hablando del proyecto de Enrique sobre un patrón con el que estaba teniendo problemas. Más adelante, se quedó solucionándolo y nos pusimos Gregorio y yo a ver el anteproyecto, después de un buen rato sacamos todos los fallos. Le explique todo lo que había visto hasta el momento sobre los excel que había visto. Me comentaron un posible patrón que sería ver si un ct produce como debe respecto a los demás de una planta porque se supone que les da el mismo sol y también comparándolo con el mismo en su media.
\section{Conclusiones}
\begin{itemize}
\item Editar el anteproyecto con los fallos que habíamos marcado.
\item Seguir viendo los excel para obtener nuevos patrones posibles.
\end{itemize}

\end{document}