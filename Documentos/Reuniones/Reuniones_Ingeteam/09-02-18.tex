\documentclass[a4paper]{article}

%% Language and font encodings
\usepackage[english]{babel}
\usepackage[utf8x]{inputenc}
\usepackage[T1]{fontenc}

%% Sets page size and margins
\usepackage[a4paper,top=3cm,bottom=2cm,left=3cm,right=3cm,marginparwidth=1.75cm]{geometry}

%% Useful packages
\usepackage{amsmath}
\usepackage{graphicx}
\usepackage[colorinlistoftodos]{todonotes}
\usepackage[colorlinks=true, allcolors=blue]{hyperref}

\begin{document}

\section{Información básica}
\textbf{Fecha y hora: }
\newline
\textbf{Participantes:}
\begin{itemize}
	\item Fernando Luján Martínez.
    \item Ignacio Garcia de Carellan Esteban-Infantes
    \item Antonio Fernández
\end{itemize}
\section{Orden del día}
\begin{itemize}
\item Ver mejor el funcionamiento de las centrales.
\item Explicación sobre algunas dudas de los excels que me habían pasado antes.
\item conocer a Ignacio, quien ocupará el puesto de Pedro.
\end{itemize}
\section{Desarrollo}
Comenzamos la reunión y Antonio nos explicó algunas cosas que había analizado a él sobre placas solares y quedamos en que pasaría el excel con el análisis. Pregunté mis dudas sobre los excels y estuvimos un rato hablando en general sobre el proyecto para que Ignacio se pusiera al día de lo que íbamos a hacer, ya que como era nuevo no tenía mucha idea.
\section{Conclusiones}
\begin{itemize}
\item Quedamos en que me pasarían los logs de fallos para poder analizar junto con los datos.
\item Ignacio se quedó con la idea de lo que estábamos haciendo.
\item Antonio me pasaría su análisis y algunos documentos sobre placas solares para el estado del arte del proyecto.
\end{itemize}

\end{document}




