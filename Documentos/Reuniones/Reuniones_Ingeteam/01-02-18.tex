\documentclass[a4paper]{article}

%% Language and font encodings
\usepackage[english]{babel}
\usepackage[utf8x]{inputenc}
\usepackage[T1]{fontenc}

%% Sets page size and margins
\usepackage[a4paper,top=3cm,bottom=2cm,left=3cm,right=3cm,marginparwidth=1.75cm]{geometry}

%% Useful packages
\usepackage{amsmath}
\usepackage{graphicx}
\usepackage[colorinlistoftodos]{todonotes}
\usepackage[colorlinks=true, allcolors=blue]{hyperref}

\begin{document}

\section{Información básica}
\textbf{Fecha y hora: 01-02-2018 12:00}
\newline
\textbf{Participantes:}
\begin{itemize}
    \item Gregorio Díaz-Descalzo
    \item Fernando Luján Martínez
    \item Vicente Requena
    \item Enrique Brazález
    \item Francisco Polo
    \item Antonio Fernández
\end{itemize}
\section{Orden del día}
\begin{itemize}
\item Conocer a nuevas personas de Ingeteam.
\item Ver quien será el nuevo responsable dentro de la empresa con el que estaremos en contacto.
\item Conocer mejor la arquitectura de su sistema que podremos hacer con los patrones.
\end{itemize}
\section{Desarrollo}
En la reunión conocimos a tres nuevas personas de la empresa, Antonio, Vicente y Francisco.
Nos comentaron que después de la marcha de Pedro (antiguo responsable) estaban esperando a la llegada del nuevo trabajador que ocupará su puesto.
Fran nos explicó como tenían montado todo el tema de eólica y que en fotovoltaica es similar. Nos indicaron que ellos tampoco saben muy bien que posibles patrones detectar pero que nos darían datos para analizar y nos ayudarían en todo lo que fuera necesario. 
\section{Conclusiones}
\begin{itemize}
\item La reunión fue esclarecedora por un lado, pues nos dejaron más claro que era lo que querían, pero, desalentadora por otro pues no tenían ni idea de patrones y supuestamente ellos son los que nos los iban a proporcionar.
\item Como persona de contacto yo tendría a Antonio.
\item Cuando llegara el nuevo haríamos una reunión él.
\end{itemize}

\end{document}